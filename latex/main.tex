\documentclass[12 pt]{scrartcl}
\usepackage{setspace}
\onehalfspacing
\usepackage{amsmath,amssymb,amsfonts,amsthm,mathtools}
\usepackage[english]{babel}
\usepackage[T1]{fontenc}
\usepackage[utf8x]{inputenc}
\usepackage{lmodern}
\usepackage{dsfont}
\usepackage{bbm}
\usepackage[round]{natbib}
\usepackage{color} 
\usepackage[defaultlines=2,all]{nowidow}
\usepackage{caption}
\usepackage[labelformat=simple]{subcaption}
\usepackage{makecell}
\renewcommand\thesubfigure{(\alph{subfigure})}

\setlength\parindent{0pt}
\setlength{\parskip}{6pt plus 1pt minus 1pt}

\newcommand{\red}{\textcolor{red}}


\begin{document}

\begin{titlepage}
  \centering
  {\scshape\LARGE TU Dortmund \par}
  \vspace{1cm}
  {\scshape\Large Introductory Case Studies \par}
  \vspace{2cm}
  {\huge\bfseries Project 3: Regression analysis for price of VW cars\par}
  \vspace{2cm}
  {\Large Lecturers:\\
    Prof.\ Dr.\ Sonja Kuhnt\\
    Dr.\ Paul Wiemann\\
    Dr.\ Birte Hellwig\\
    M.\ Sc.\ Hendrik Dohme \par}
  \vspace{1cm}
  {\Large Author: Tadeo Hepperle \par}
  \vspace{0.5 cm}
  {\Large Group number: 2\par}
  \vspace{0.5 cm}
  {\Large Group members: Prem Kant Shekhar, Minjae Ok, Ishan Singh Dhapola, Fyalisia Amanda Putri, Tadeo Hepperle}
  \vfill
  {\large \today\par}
\end{titlepage}


\tableofcontents

\cleardoublepage

\section{Introduction}


\section{Problem statement}

We analyzed a dataset containing prices and other features of 438 cars from the german car brand VW to determine which factors influence the price of a car and to what extent they do so. This makes it possible to predict the value of a car by knowledge about its features.

\subsection{Description of the dataset}

The dataset used in this report is a sclice of a bigger dataset from kaggle.com and contains data about 438 VW cars, which were originally scraped in 2020 from the british car internet platform "Exchange and Mart" \citep{Exchangeandmart} (https://www.exchangeandmart.co.uk/). The slicing was done in a way to only include the VW models "Passat", "T-Roc" and "Up".
For each of the 438 cars information on the following variables is provided:
\begin{itemize}
  \item the \textbf{price} a car is selling for in 1000 GBP (£), metric variable
  \item the \textbf{year} a car was first registered, metric variable
  \item the \textbf{mileage} as the total distance a car has been driven in 1000 miles, metric variable
  \item the \textbf{fuel consumption} in mpg, the number of miles a car can drive with one gallon of fuel, metric variable
  \item the \textbf{fuel type} a car uses, categorical variable with 3 levels: "Diesel", "Hybrid", "Petrol"
  \item the \textbf{engine size} as the volume of fuel in liters and air a car can fit in its engines cylinders, metric variable
  \item the annual \textbf{tax}, also known as "Vehicle Excise Duty" that has to be paid for the car annually
  \item the type of \textbf{transmission} a car has, categorical variable with 3 levels: "Manual", "Semi-Auto", "Automatic"
  \item \item the \textbf{model} of the car, categorical variable with 3 levels: "Passat", "T-Roc", "Up"
\end{itemize}
From those 9 original variables, three more are computed:
\begin{itemize}
  \item the \textbf{logprice} as the natural logarithm of the \textbf{price} of a car
  \item the \textbf{fuel consumption} in liters per 100 kilometers, computed from the fuel consumption in mpg as $\frac{282.48}{mpg}$
  \item the \textbf{age} of the car in 2020 when the data was taken, computed as $2020-year$
\end{itemize}
From here on, when fuel consumption is mentioned, the liters per 100 kilometers value is meant.
The quality of the data seems good, "Exchange and Mart" is a large and well known site and no data is missing. Although we have to rely on the data given to us and cannot check if for example the mileage and tax are truely correct.

\subsection{Objectives of the report}
The goal of this report is to predict the price of a car as accurately as possible. To achieve this we fit a multitude of linear regression models and using best subset selection will choose as the final model the one with the best indicators, namely the BIC (Bayesian information criterion). Interactions between the 8 predictors or nonliner relationships will not be considered. \\
Also we will figure out if it makes more sense to predict the price directly through a linear regression model or if it is better to use the logprice as the dependent variable. After making predictions this can be retransformed to normal prices so no information is lost, but it could be that the logarithmic price fits the data better.
Sadly the range of car models in the data is quite limited and therefore we will only be able to predict prices of those models later on with our regression models.

\section{Statistical Methods}

We give a brief overview about the math behind multiple linear regression, how to select the best model in subset selection via certain indicators and how to assess how good a regression model fits the data.

\subsection{classical linear regression}

Multiple linear regression is a method for predicting a metric variable $y$ on the basis of $k$ metric variables $x_1, \dots, x_k$. A linear regression model consists of kcoefficients that have to be fitted to the data and can be represented by the following formula:
\[ y = \beta_1x_1 + \dots + \beta_kx_k + \epsilon\]
y is also called depentant variable or outcome, while  $x_1$ to $x_k$ are called the independent variables or predictors. Since the coefficients $\beta_1, \dots, \beta_k$ are just multiplied by the predictors $x_1, \dots, x_k$, a linear regression can only detect linear relationships in the data. In the formula $\epsilon$ represents the error term, because it is likely that even the best linear combination of the multiple $\beta_jx_j$ will not add up to the real value of $y$.
But what do we mean by "the best linear combination"? Typically the coefficients in linear regression are estimated by minimizing the residual square sum (RSS). If we have all $x_1$ to $x_k$  and $\beta_1$ to $\beta_k$ we can estimate $y$ as $\hat{y}$:
\[ \hat{y} = \beta_1x_1 + \dots + \beta_kx_k\]
The error term $\epsilon$ is missing in this equation. $y = \hat{y} + \epsilon$ or $\epsilon = y - \hat{y}$.
Suppose we have a data set consisting of $n$ objects $i$ with values $y_i$ and $x_{1,i}$ to $x_{k,i}$. Each object has its own error term $\epsilon_i$, sometimes we overestimate $y$ with $\hat{y}$, sometimes we underestimate it. After fitting the model coefficients by minimizing the RSS, the mean error ($\sum^{n}_{i=1}{\epsilon_i}$) will be zero.
\[RSS = \sum^{n}_{i=1}{(y_i - \hat{y_i})^2} = \sum^{n}_{i=1}{(\epsilon_i)^2} \]
To estimate the coefficients, we  introduce some matrix notation. Let $Y$ be a $n \times 1$ column vector containing $y_1$ to $y_n$. X is a $n \times k$-matrix consisting of n row vectors that contain the values $x_{1,i}$ to $x_{k,i}$ for each object. Then we can define $\mathcal{B}$ as a $k \times 1$ vector containing the coefficients  $\beta_1, \dots, \beta_k$ and we define $\mathcal{E}$ as a $n \times 1$ column vector containing $\epsilon_1$ to $\epsilon_n$.
In this setting we can then write the formula as the following:
\[ Y =  X\mathcal{B} + \mathcal{E}\]
Here the RSS depends on $\mathcal{B}$ can be rewritten as follows \citep[p.~105]{fahrmeir2013regression}:
\begin{equation} \label{eq2}
  \begin{split}
    RSS(\mathcal{B}) & = \mathcal{E}^T\mathcal{E} \\
    & = (Y-X\mathcal{B})^T(Y-X\mathcal{B}) \\
    & = Y^TY - B^TX^TY - Y^TX\mathcal{B} + \mathcal{B}^TX^TX\mathcal{B} \\
    & = Y^TY   - 2Y^TX\mathcal{B} + \mathcal{B}^TX^TX\mathcal{B} \\
  \end{split}
\end{equation}
Taking the derivative with respect to $\mathcal{B}$ now yields:
\[ \frac{\partial}{\partial \mathcal{B}} RSS(\mathcal{B}) = -2X^TY + 2X^TX\mathcal{B}\]
It can be shown that the second derivative is positive and therefore we can find a minimum for $\mathcal{B} = \hat{\mathcal{B}}$ by setting the first derivative to zero \citep[p.~106]{fahrmeir2013regression}.$\hat{\mathcal{B}}$ represents the least squares estimator for $\hat{\mathcal{B}}$.
\begin{equation} \label{eq3}
  \begin{split}
    -2X^TY + 2X^TX\hat{\mathcal{B}} & = 0 \\
    2X^TX\hat{\mathcal{B}}    & =2X^TY \\
    \hat{\mathcal{B}}  & =(X^TX)^{-1}X^TY
  \end{split}
\end{equation}
So this is how we calculate our estimated coefficients $\hat{\beta}_1, \dots, \hat{\beta}_k$ as entries in the vector $\hat{\mathcal{B}}$. This classical linear model is missing an intercept though. Setting all predictors to zero would always yield $\hat{y} = 0$, no matter the coefficients. This is obviously not an optimal estimate. To accomodate for this, often a predictor $x_0$ is introduced that gets a value of 1 assigned for each object in the data. As a part of the predictor matrix $X$ the method described above will then estimate a coefficient $\hat{\beta_0}$, known as the intercept.

A coefficient $\hat{\beta}_i$ can be interpreted in the following way: Holding all other predictor variables constant, an increase of one unit in $x_i$ will on average result in an increase of $\hat{y}$ by$\hat{\beta}_i$. The intercept $\beta_0$ represents the expected value of $y$ when $x_1 = \dots = x_k = 0$.
Sometimes logarithmic transformations are applied to the variables before fed into in the linear regression. When the outcome $y$ is actually a proxy for $ln(y_{original})$, than a one unit increase in $x_i$ leads to a $\hat{\beta}_i$ increase in $y$ which means a multiplication of $y_{original}$ by $10^{\hat{\beta}_i}$. For small $\hat{\beta}_i$ this means: a one unit increase in $x_i$ leads to a $100 \cdot \hat{\beta}_i \%$ increase in $y_{original}$.

To assess how good a model is, a statistic $R^2$ can be calculated. It represents how much variance in $y$ can be explained by $x_1, \dots , x_k$, or in other words how much variance $y$ shares with $\hat{y}$. Therefore it can be calculated as 1 minus the fraction between RSS and TSS where TSS is the empirical variance of y multiplied by n \citep[p.~234]{james2013introduction}.
\begin{equation} \label{eqrsquared}
  \begin{split}
    R^2  & = 1-\frac{RSS}{TSS} \\
    & = 1-\frac{RSS}{\sum{(y_i - \overline{y})^2}} \\
    & = 1-\frac{\sum{(y_i - \hat{y}_i)^2}}{\sum{(y_i - \overline{y})^2}} \\
    & = \frac{\sum{(\hat{y}_i -\overline{y})^2}}{\sum{(y_i - \overline{y})^2}}
  \end{split}
\end{equation}

It can also be calculated as the squared correlation between the real outcome $y$ and the predicted outcome $\hat{y}$ \citep[p.~113]{fahrmeir2013regression}.
\[ R^2 =  r_{y\hat{y}}^2\]
Therefore $R^2$ can take values between 0 and 1. A greater $R^2$ means a more accurate prediction. What a good $R^2$ value is, highly depends on the application though, while in physics often values close to 1 can be found, in "biology, psychology, marketing,and other domains" values below $R^2 \approx 0.10$ are to be expected according to \citet[p.~70]{james2013introduction}.

A linear regression should only be used if some assumptions are fulfilled. First, the number of observations should be greater than the number of predictors. The residuals should be independent of the predicted values. For this, a scatter plot for $\hat{y}$ vs. $\epsilon$ should not show a specific pattern. Their variance should not change as predicted values increase or descrease. This is also known as equal variance or homoscedasticity \citep{10Assump62}. Also the correlation between residuals and each predictor should be zero. This however is fulfilled when minimizing the RSS, because otherwise, the predictors "know" something about the residuals, which means they know more about y and are therefore not optimal. In addition to that the residuals should be approximately normally distributed which can be checked by looking at a QQ-plot between quantiles of standardized residuals and the standard-normal distribution \citep{10Assump62}.

\subsection{dummy coding of categorical variables}
The question arises how to use categorical variables as predictors in multiple linear regression. This can be resolved by a coding system. A coding system creates for every categorical variable a number of derived metric variables that can then be used in linear regression. One such coding system is dummy coding. citeXXXXX
Assuming we have a categorical variable $x_i$ with $m \ge 2$ levels with labels $l_1, \dots, l_m$. We can then create a dichotomous variable $x_{l_j}$ for each label $l_j \in\{l_2, \dots, l_m\}$, such that:
\[
  x_{l_j}=
  \begin{cases}
    1, & \text{if } x_i = l_j \\
    0, & \text{otherwise}
  \end{cases}
\]
In this way all information about the categorical variable $x_i$ is contained in the variables $x_{l_2}, \dots, x_{l_m}$. A datapoint with label $x_i = l_1$ can be recognized by having $x_{l_2} = \dots = x_{l_m} = 0$, $l_1$ is also called the baseline accoring to \citet[p.~86]{james2013introduction}. Those newly created numerical variables can then be used in regression analysis as predictors.

\subsection{model diagnostics}
Assessing how well a linear model fits the data, is aided by some graphical and numerical methods.
First, we can look at $R^2$ as a measure of goodness of fit. Also it is important to check the assumptions of a linear regression. For this some plots can help \citep{Understa39}):
\begin{itemize}
  \item Q-Q-plot of standardized residuals and a standard normal distributiono check if residuals are approximately normally distributed
  \item Q-Q-plot of $y$ and $\hat{y}$ to check if they follow approximately the same distribution
  \item scale-location-plot: a scatter plot of $\hat{y}$ on the x-axis and standardized residuals on the y-axis to see if they are evenly distributed along $\hat{y}$ .
\end{itemize}

\subsection{best subset selection}

The more predictors go into a linear regression model, the more accurate it will perform on the training data and the greater its $R^2$ value will be. But taking more predictors into account is not always useful. It makes the model harder to interpret and does not highlight which variables are actually important for the outcome. Moreover a lot of times variance in $y$ that can be explained by an additional predictor is already explained by other predictors, such that the gain in variance explanation through $R^2$ is marginal and could even just be a product of overfitting XXXXjameschapter2:Rsquared on train data is higher than on test data.
To determine which predictors should actually be taken into account there is a method called "best subset selection" as described by \citet[p.~227]{james2013introduction}.

To perform best subset selection on data with k predictors, for each $i \in {0,1,...,k}$ all possible linear models with a number of exactly $i$ predictors are computed. There are ${k}\choose{i}$ different models for each $i$. Among those models for every $i$ the model with least RSS is selected. This results in k + 1 models (including the empty model) that need to be considered to choose the best model. Finally from those  k + 1 models the one with the best score on some indicator is chosen.
possible indicators can be:
\subsubsection{adjusted $R^2$}

While $R^2$ explaines how much variance in the training data can be explained by the predictors it is an overestimation for the performance on actual test data. Also it gets only larger with more predictors which might be a problem. Therefore an $Adjusted R^2$ can be calculated \citep[p.~234]{james2013introduction}.
\[Adjusted R^2 = 1- \frac{RSS/(n-k-1)}{TSS(n-1)}\]
The term above the division line contains k and punishes models with more predictors. When used as an indicator in best subset selection the model with the greatest adjusted $R^2$ should be chosen.

\subsubsection{Akaike information criterion (AIC) }
According to \citet[p.~234]{james2013introduction} the AIC can be calculated using the following formula, where $k$ denotes the number of predictors and $\hat{\sigma}^2$ is  the variance of $\epsilon$ as $VAR(y-\hat{y})$ when looking at the model using all available predictors.
\[AIC = \frac{1}{n}(RSS + 2k\hat{\sigma}^2)\]
Please note that for simplicity normalizing constants have been left out as they do not matter when comparing two models \citep[p.~234]{james2013introduction}.
A better model is characterized by a lower AIC.

\subsubsection{Bayesian information criterion (BIC) }

The Bayesian information criterion can be computed quite similarly to the AIC:
\[BIC = \frac{1}{n}(RSS + ln(n)k\hat{\sigma}^2)\]
A low BIC characterizes a good model. The only difference is that the BIC uses the natural logarithm of n in the formula instead of $2$. Because $ln(n) > 2$ for $n \ge 8$ and we usually have more than 8 datapoints in our data, the BIC penalizes more predictors more heavily than the AIC and will therefore result in models with less predictors compared to the AIC when used in best subset selection \citep[p.~234]{james2013introduction}.

\subsection{inference and linear models}

To check if a linear regression model with k predictors can predict a significant portion of the variance in the dependent variable an F statistic can be calculated \citep[p.~76]{james2013introduction}. Under the assumtion that the H0 (=model cannot predict variance in outcome) is true, $F$ follows an $F_{a,b}$ distribution with $a = n-k$ and $b = k-1$.
\[F = \frac{(TSS-RSS)/k}{RSS/(n-k-1)}\]
Each coefficient $\hat{\beta}_j, j \in {1,...,k}$ can also be tested for statistical significance. For this, according to \citet[p.~67]{james2013introduction},we can compute a t statistic $t_j$ by dividing the difference between $\hat{\beta}_j$ and the coefficient assumed under the null hypothesis ($\beta_{H_0}$) by the standard error of $\hat{\beta}_j$:
\[ t_j = \frac{\hat{\beta}_j - \beta_{H_0}}{SE(\hat{\beta}_j)} \]
Under $H_0$ this would follow a t-distribution with n-k-1 degrees of freedom, which can then be compared to the 1-$\frac{1}{2}\alpha$-quantile of said distribution as a critical value.
The standard error $SE(\hat{\beta}_j)$ can be computed with the following formula \citep[p.~66]{james2013introduction}, where $\hat{\sigma}^2$ is the residual standard error that can be calculated from the residuals.
\[SE(\hat{\beta}_j) = \sqrt{\frac{\hat{\sigma}^2}{\sum_{i=1}^n{(x_i-\overline{x})^2}}}\]
\[ \hat{\sigma}^2 = \sqrt{RSS/(n-1-k)}  \]
This also allows for computing a 1-$\alpha$-confidence interval for each $\hat{\beta}_j$, that signify that $\hat{\beta}_j$ differs significanctly from 0 if they do not contain the 0. a 1-$\alpha$-confidence interval for a parameter $\hat{\beta}_j$ has the property, that if we would take many random samples from the same population, 1-$\alpha\cdot100\%$ of confidence intervals will contain the true unknown parameter $\beta_j$ \citep[p.~66]{james2013introduction}.

\section{Results}

\subsection{Data Transformation}

\subsection{Descriptive Statistics}

\subsection{Determining the dependent variable}


\section{Summary}


\newpage
\addcontentsline{toc}{section}{Bibliography}
\renewcommand\refname{Bibliography}
\bibliographystyle{plainnat}
\bibliography{references}
\end{document}
